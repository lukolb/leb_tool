\documentclass[12pt,a4paper]{article}
\usepackage[utf8]{inputenc}
\usepackage[T1]{fontenc}
\usepackage[ngerman]{babel}
\usepackage{graphicx}
\usepackage{geometry}
\usepackage{hyperref}
\usepackage{float}
\usepackage{xcolor}
\usepackage{enumitem}
\usepackage{array}
\usepackage{tabularx}
\geometry{margin=2.5cm}
\setlength{\parskip}{6pt}
\setlength{\parindent}{0pt}

\definecolor{primary}{HTML}{0B57D0}
\definecolor{secondary}{HTML}{111111}
\hypersetup{
  colorlinks=true,
  linkcolor=primary,
  urlcolor=primary,
  citecolor=primary
}

\begin{document}

% Titelblatt
\pagenumbering{gobble}
\begin{titlepage}
  \centering
  \vspace*{1cm}
  \includegraphics[width=4cm]{../assets/icons/lebtool-icon-512x512.png}\\[1cm]
  {\LARGE \textbf{LEB Tool}}\\[0.2cm]
  {\Large Digitale Lernentwicklungsberichte}\\[1cm]
  \rule{\textwidth}{0.4pt}\\[0.8cm]
  {\Large \textbf{Benutzerhandbuch}}\\[0.5cm]
  {\large Version nach aktuellem Entwicklungsstand}\\[2cm]
  \vfill
  {\large Für Administrator:innen, Lehrkräfte, Schüler:innen und Eltern}\par
  \vspace*{1.5cm}
\end{titlepage}

\tableofcontents
\newpage
\pagenumbering{arabic}

\section{Einleitung}
Das LEB Tool ist eine webbasierte Anwendung für die strukturierte und datenschutzkonforme Erstellung von Lernentwicklungsberichten im Grundschulkontext. Dieses Handbuch beschreibt Installation, Einrichtung und Nutzung der unterschiedlichen Rollen (Admin, Lehrkraft, Schüler:in, Eltern) und basiert auf den aktuell im Code verfügbaren Funktionen.

\subsection{Leistungsmerkmale im Überblick}
\begin{itemize}
  \item Rollenbasiertes System mit getrennten Oberflächen und Rechten
  \item Delegationen für Teamarbeit nach Klasse\,\(\times\,\)Fachbereich
  \item Fortschrittslogik zur Vollständigkeitsprüfung in UI und Exporten
  \item Ausfüllbare PDF-Formulare mit Platzhaltersystem
  \item Audit-Log für revisionssichere Nachvollziehbarkeit
  \item Optionale KI-Unterstützung für Textvorschläge (konfigurierbar)
  \item QR-basierter, passwortloser Schülerzugang mit Vorlesefunktion (TTS)
  \item Eltern-Feedback mit eigenem Routing und CSRF-Schutz
\end{itemize}

\section{Systemüberblick}
\subsection{Zielsetzung}
\begin{itemize}
  \item Einheitliche und nachvollziehbare Lernentwicklungsberichte
  \item Reduktion von Copy\&Paste und manuellen Fehlern
  \item Klare Rollen- und Rechteverteilung
  \item Zusammenarbeit mehrerer Lehrkräfte durch Delegationen
  \item Transparenz durch Audit-Logging
  \item DSGVO-konforme Datenhaltung und Löschung
\end{itemize}

\subsection{Hauptmodule}
\begin{description}[leftmargin=1.6cm]
  \item[Admin (/admin)] Vollständiger Systemzugriff mit Verwaltung von Personen, Klassen, Templates und globalen Einstellungen.
  \item[Lehrkraft (/teacher)] Klassenbezogene Arbeit, Datenerfassung, Delegationen und PDF-Export.
  \item[Schüler:in (/student)] Passwortloser Zugriff per QR-Code für Selbsteinschätzungen.
  \item[Eltern (/parent)] Optionales Feedback-Formular, klar getrennt vom übrigen System.
\end{description}

\section{Installation}
Das System ist für klassisches Shared-Webhosting ohne Shell-Zugriff ausgelegt. Die folgenden Schritte beziehen sich auf den bereitgestellten Code.

\subsection{Voraussetzungen}
\begin{itemize}
  \item PHP 8.1+ mit PDO-MySQL-Erweiterung
  \item MySQL/MariaDB-Datenbank (Zeichensatz \texttt{utf8mb4})
  \item HTTPS-fähiges Hosting (empfohlen) mit Session-Unterstützung
  \item Schreibrechte für \texttt{uploads/} und temporäre Dateien
  \item Mailversand (SMTP oder serverseitig), falls Passwort-Reset per E-Mail genutzt wird
\end{itemize}

\begin{enumerate}
  \item \textbf{Dateien kopieren:} Alle Projektdateien auf den Webserver hochladen.
  \item \textbf{Installationsassistent starten:} \texttt{install.php} im Browser aufrufen.
  \item \textbf{Datenbank konfigurieren:} Verbindungsdaten angeben (Host, Port, Name, Benutzer, Passwort, Zeichensatz). Vorgeschlagenes Charset ist \texttt{utf8mb4} (siehe \texttt{config.sample.php}).
  \item \textbf{Applikationspfade setzen:} \texttt{base\_path} (relativer Pfad zur Installation) und \texttt{public\_base\_url} (vollständige URL) prüfen; bei Subfolder-Installationen \texttt{base\_path} entsprechend setzen.
  \item \textbf{Branding festlegen:} Primärfarbe, Sekundärfarbe und Logo-Pfad definieren (optional bereits im Installer verfügbar). Das mitgelieferte Icon befindet sich unter \texttt{assets/icons/lebtool-icon-512x512.png}.
  \item \textbf{Admin-Account anlegen:} Zugangsdaten im Installer vergeben (Name, E-Mail, Passwort); die Session-Kennung kann bei Bedarf in \texttt{app.session\_name} geändert werden.
  \item \textbf{Abschluss und Absicherung:} Installation abschließen und \texttt{install.php} anschließend löschen oder umbenennen; \texttt{bootstrap.php} bleibt als zentraler Einstieg bestehen.
  \item \textbf{Konfigurationsdatei prüfen:} \texttt{config.sample.php} nach \texttt{config.php} kopieren und anpassen (z.\,B. KI-Provider, Mail-Absender, Upload-Verzeichnis, Standard-Schuljahr, Elternsicht-Flag, PDF-Pfade, TTS-Stimme).
\end{enumerate}

\subsection{Update-Hinweise}
\begin{itemize}
  \item Vor Updates stets Datenbank- und Datei-Backup erstellen.
  \item Installer nach Updates erneut entfernen/absichern.
  \item Individuelle Konfiguration in \texttt{config.php} nicht überschreiben.
  \item Browser-Cache leeren, falls Styles nicht aktualisiert werden.
\end{itemize}

\section{Erstkonfiguration}
\subsection{Systemparameter}
\begin{itemize}
  \item \textbf{Branding:} Farben und Logo für Schulauftritt setzen (\texttt{app.brand}).
  \item \textbf{Uploads:} Verzeichnis \texttt{uploads} mit Schreibrechten bereitstellen.
  \item \textbf{Mail:} Absenderadresse in \texttt{mail.from\_email} und \texttt{mail.from\_name} festlegen.
  \item \textbf{KI-Integration:} In \texttt{ai} API-Key und Modell konfigurieren (optional). Buttons erscheinen nur bei aktiver Konfiguration.
  \item \textbf{Vorleseoptionen (Schülerbereich):} Standard-Stimme und Geschwindigkeit in \texttt{student.tts\_voice} und \texttt{student.tts\_rate} anpassen.
  \item \textbf{Schuljahr-Voreinstellung:} \texttt{app.default\_school\_year} für CSV-Importe ohne Jahresangabe.
\end{itemize}

\subsection{Benutzer und Klassen anlegen}
\begin{enumerate}
  \item Als Admin einloggen und Lehrkräfte erfassen.
  \item Klassen anlegen (aktiv/archiviert) und Schüler hinzufügen.
  \item Schüler den Klassen zuordnen; Delegationsregeln definieren.
  \item Templates und Template-Felder konfigurieren (Text, Option, Datum, Systembindung, Gruppen, Filter).
  \item Feature-Flags und globale Einstellungen nach Bedarf setzen.
\end{enumerate}

\subsection{Benutzerkonten und Passwörter}
\begin{itemize}
  \item Passwortrücksetzung über \texttt{forgot\_password.php}; neue Passwörter werden nach Bestätigung gesetzt.
  \item Passwortänderung im eingeloggten Zustand über \texttt{changepassword.php}.
  \item Admins können Konten neu setzen oder deaktivieren; Protokollierung erfolgt im Audit-Log.
\end{itemize}

\section{Rollenbasierte Nutzung}
\subsection{Administrator:in}
\textbf{Zugriff:} \url{.../admin}

\textbf{Zentrale Funktionen}
\begin{itemize}
  \item Lehrkräfte- und Klassenverwaltung (aktiv/archiviert) mit Suche, Sortierung und CSV-Import von Schülern
  \item Schülerverwaltung und Zuordnung zu Klassen, inkl. QR-Tokens neu generieren
  \item Template-Management (Vorlagen, Felder, Optionslisten, Gruppen, Filterbarkeit, Pflichtstatus)
  \item Globale Einstellungen und Feature-Flags (z.\,B. KI, Elternsicht, Klassenexport, Branding)
  \item Audit-Log mit Filtern (User, Event, Zeitraum), Pagination, Sortierung und optionaler IP-Anzeige
  \item DSGVO-konformes Löschen personenbezogener Daten (Hard-Delete) mit Log-Eintrag
\end{itemize}

\textbf{Besonderheiten}
\begin{itemize}
  \item Zugriff auf alle Klassen und Delegationen
  \item Vollständige Protokollierung aller Admin-Aktionen
  \item Export-API zugänglich (PDF mit Platzhaltern wie \texttt{{\{\}student.firstname\}})
\end{itemize}

\textbf{Wichtige Bereiche}
\begin{description}[leftmargin=1.6cm]
  \item[Lehrkräfte] Anlegen, Bearbeiten, Aktivieren/Deaktivieren, Passwort-Reset, Suchfeld nach Name/E-Mail.
  \item[Klassen] Erstellen/Archivieren mit Schuljahr und Stufe, Template-Zuordnung, QR-Reissue, TTS-Freigabe, Delegationssicht.
  \item[Schüler:innen] Stammdaten, Klassenwechsel, QR-Token-Reset, Delegationsprüfung, Fortschrittsanzeige pro Klasse.
  \item[Templates] Gruppenreihenfolge, Feldtypen (Text/Option/Datum/System), Pflicht- und Filter-Flags, Optionslisten inkl. Vorlagen.
  \item[Einstellungen] Basis-URLs, Branding, Upload-Pfade, KI-Provider/Modell, Elternsicht-Flag, Klassenexport-Flag, Session-Name.
  \item[Audit-Log] Detailansicht mit JSON-Daten; Filter auf Benutzer, Event, Zeitraum; ID-Auflösung zu Klartext; Export via Browser-Druck.
\end{description}

\textbf{Arbeitsbereiche im Detail}
\begin{description}[leftmargin=1.6cm]
  \item[Dashboard] Schnellzugriff auf Klassenstatus, offene Delegationen und zuletzt exportierte PDFs.
  \item[Lehrkräfte] Anlegen, Bearbeiten, Deaktivieren, Passwort-Reset; Suche und Sortierung nach Namen/E-Mail.
  \item[Klassen] Erstellen/Archivieren, Schuljahr und Stufe pflegen, Template zuweisen, TTS-Freigabe setzen, QR-Codes für Schüler neu generieren.
  \item[Schüler:innen] Stammdaten verwalten, Klassenwechsel durchführen, QR-Tokens neu ausstellen, Zugehörigkeit zu Delegationen prüfen.
  \item[Templates] Vorlagen strukturieren (Gruppen, Reihenfolge), Felder anlegen (Typ Text/Option/Datum/System), Pflicht- und Filter-Flags setzen, Optionslisten pflegen (inkl. Vorlagen).
  \item[Einstellungen] Branding (Farben/Logo), Basis-URLs, Upload-Pfade, KI-Provider/Modell, Feature-Flags für Elternsicht und Klassenexport.
  \item[Audit-Log] Detailansicht mit JSON-Daten; Filter auf Benutzer, Event, Zeitfenster; CSV/PDF-Export via Standard-Print.
\end{description}

\subsection{Lehrkraft}
\textbf{Zugriff:} \url{.../teacher}

\textbf{Arbeitsablauf}
\begin{enumerate}
  \item Anmeldung und Auswahl einer Klasse oder eines delegierten Fachbereichs.
  \item Schülerdaten prüfen und Lernentwicklungsfelder ausfüllen (Pflichtfelder werden berücksichtigt, systemgebundene Felder werden ignoriert).
  \item Delegationen verwalten: Fachbereiche an Kolleg:innen delegieren, Status einsehen oder ändern.
  \item Live-Vorschau nutzen und PDF-Exporte erstellen (einzelner Schüler oder, je nach Konfiguration, Klassenexport).
  \item Fortschrittsanzeigen beachten (fehlende Felder, Vollständigkeit) und offene Schüler- oder Lehrerfelder abschließen.
  \item KI-Vorschläge anstoßen, Text prüfen und gezielt übernehmen; Speicherung erfolgt erst nach manueller Bestätigung.
\end{enumerate}

\textbf{Werkzeuge}
\begin{itemize}
  \item Filter und Suche innerhalb von Klassen
  \item Fortschrittslogik für Berichte (Pflichtfelder, offene Abschnitte)
  \item KI-Textvorschläge (falls aktiviert) ohne automatische Speicherung, Timeout konfigurierbar
  \item Delegationsübersicht (eigene Klassen, delegierte Klassen, delegierte Fächer)
  \item PDF-Export mit Platzhalterauflösung; Auswahl einzelner Schüler oder gesamter Klasse (abhängig von Einstellung)
  \item Klassen- und Schülerlisten mit Sortierung/Filterung
\end{itemize}

\textbf{Formular- und Feldtypen}
\begin{itemize}
  \item Freitextfelder mit Rich-Text-ähnlicher Eingabe (Absätze, Listen)
  \item Optionsfelder (Drop-down) inkl. Optionslisten-Vorlagen
  \item Datumsfelder mit Formatierung nach Templatevorgabe
  \item Systemfelder werden angezeigt, sind aber nicht bearbeitbar und fließen in Exporte ein
\end{itemize}

\textbf{Delegationen}\footnote{Delegationen betreffen immer Klasse \(\times\) Fachbereich.}
\begin{itemize}
  \item Statuspfad: offen \(\rightarrow\) in Bearbeitung \(\rightarrow\) abgeschlossen (zurücksetzbar)
  \item Lehrkräfte können eigene Delegationen abgeben oder zurückholen; Admins können jederzeit eingreifen
  \item Delegierte Bereiche erscheinen separat in der Übersicht
\end{itemize}

\textbf{Speicher- und Vorschauflogik}
\begin{itemize}
  \item Änderungen werden serverseitig gespeichert; Statusanzeigen zeigen \enquote{Speichern}, \enquote{Gespeichert} oder Fehler an; fehlende Pflichtfelder werden markiert.
  \item Live-Vorschau nutzt aktuelle Eingaben und Template-Placeholders; Export ist identisch mit der Vorschau gerendert.
\end{itemize}

\subsection{Schüler:in}
\textbf{Zugriff:} \url{.../student} (passwortlos)

\textbf{Merkmale}
\begin{itemize}
  \item Login per QR-Code; kein Benutzername/Passwort notwendig
  \item Nur freigegebene Felder sichtbar
  \item Automatisches Speichern mit Statusanzeige (Speichern/OK/Fehler); Fortschritt pro Schritt und Gesamtreport
  \item Tokenbasierter Login verhindert Fremdzugriff; Lehrer können Tokens neu generieren
  \item TTS-Bar mit Play-/Stop-Button; Stimme und Geschwindigkeit konfigurierbar (\texttt{student.tts\_voice}, \texttt{student.tts\_rate})
\end{itemize}

\subsection{Eltern}
\textbf{Zugriff:} \url{.../parent}

\textbf{Aktueller Stand}
\begin{itemize}
  \item Eltern-Feedback-Formular mit CSRF-Schutz
  \item Eigenes Routing, keine Einsicht in Verwaltungs- oder Schülerdaten
  \item Bestätigte Abgabe wird im Audit-Log vermerkt; Rückmeldungen werden getrennt gespeichert
\end{itemize}

\textbf{Geplante Erweiterungen (bereits vorbereitet)}
\begin{itemize}
  \item Elternansicht der Berichte
  \item Unterschriftenfeld mit Lehrkraftname (nur Elternansicht)
\end{itemize}

\section{PDF- und Exportfunktionen}
\begin{itemize}
  \item Unterstützung ausfüllbarer PDF-Formulare (AcroForms)
  \item Platzhalter wie \texttt{{\{\}student.firstname\}}, \texttt{{\{\}class.label\}}, \texttt{{\{\}teacher.fullname\}} und formatierbare Systemfelder (Datum/Zeit)
  \item Einheitliche Export-API unter \texttt{/shared/export\_*} (z.\,B. \texttt{export\_class.php}, \texttt{export\_student.php})
  \item Zugriffskontrolle: Lehrkräfte nur auf eigene Klassen, Admin auf alle Daten; Delegationen werden berücksichtigt
  \item Klassen- oder Einzel-Exports je nach Einstellung; Dateiablage in \texttt{uploads/} (konfigurierbar)
  \item Export-Historie sichtbar über Browser-Downloads; Dateinamen enthalten Klasse, Schüler und Zeitstempel
  \item Template-Abhängigkeiten: Pflichtfelder und Optionslisten fließen automatisch ein
\end{itemize}

\section{Delegationen und Zusammenarbeit}
\begin{itemize}
  \item Delegation pro Kombination aus Klasse und Fachbereich
  \item Statusverfolgung (offen / in Bearbeitung / abgeschlossen)
  \item Delegationen können geändert oder widerrufen werden
  \item Darstellung delegierter sowie delegierender Klassen
  \item Admin kann jederzeit eingreifen
\end{itemize}

\paragraph{Best Practices}
\begin{itemize}
  \item Delegationen früh setzen, damit Fortschrittslogik pro Fachbereich greift.
  \item Bei Rücknahme einer Delegation den Status auf \enquote{offen} setzen, um offene Pflichtfelder sichtbar zu machen.
  \item QR-Tokens nach Delegationswechsel neu ausstellen, falls Schülerfelder betroffen sind.
\end{itemize}

\section{Fortschritts- und Vollständigkeitslogik}
\begin{itemize}
  \item Automatische Berechnung des Bearbeitungsstands pro Schüler und Klasse
  \item Berücksichtigung fehlender Pflichtfelder und abgeschlossener Eingaben
  \item Anzeige in Klassenübersichten, Lehrkraft-UI und Exporten
  \item Wizard-Navigation im Schülerbereich mit Status pro Schritt
\end{itemize}

\section{Datenschutz und Sicherheit}
\begin{itemize}
  \item Rollenbasierte Zugriffskontrolle
  \item CSRF-Schutz für schreibende Aktionen
  \item QR-Token statt Passwörter für Schülerzugänge
  \item Trennung von Stammdaten und Berichtsdaten
  \item Audit-Log für Nachvollziehbarkeit
  \item Empfohlen: HTTPS erzwingen, regelmäßige Backups, Entfernung von \texttt{install.php} nach Setup
\end{itemize}

\subsection{Audit-Log im Detail}
\begin{itemize}
  \item Speichert Benutzer, Aktion, Zeitstempel, Entitäts-IDs und strukturierte JSON-Daten.
  \item Filterbar nach Benutzer, Ereignistyp und Zeitraum; sortierbar und paginiert.
  \item Optional einblendbare IP-Adresse zur Nachvollziehbarkeit (konfigurierbar).
  \item Aktionen wie Login, Passwortänderung, Delegationswechsel, Export und Löschung werden protokolliert.
\end{itemize}

\subsection{KI-Unterstützung}
\begin{itemize}
  \item Aktivierung über \texttt{ai.enabled}; Modell und API-Key in \texttt{ai.model} und \texttt{ai.api\_key} hinterlegen.
  \item Schaltflächen erscheinen nur für Felder, die Textvorschläge unterstützen; Ergebnis muss manuell übernommen werden.
  \item Timeout und maximale Tokens in der Konfiguration anpassbar; keine automatischen externen Aufrufe ohne Opt-in.
  \item Generierte Texte werden nicht gespeichert, bevor die Lehrkraft sie explizit bestätigt.
\end{itemize}

\section{Fehlerbehebung und Tipps}
\begin{itemize}
  \item \textbf{Login-Probleme:} Cookies und Session-Namen prüfen (\texttt{app.session\_name}).
  \item \textbf{PDF-Export fehlschlägt:} Stellen Sie sicher, dass Template-Platzhalter vorhanden und Felder vollständig sind.
  \item \textbf{KI-Vorschläge erscheinen nicht:} \texttt{ai.enabled} und API-Key prüfen.
  \item \textbf{Uploads schlagen fehl:} Schreibrechte für \texttt{uploads}-Verzeichnis setzen.
  \item \textbf{QR-Code ungültig:} Token im Admin- oder Lehrkräftebereich neu generieren und erneut verteilen.
  \item \textbf{Langsame Antwortzeiten:} PHP-Timeout und KI-Timeout (\texttt{ai.timeout\_seconds}) kontrollieren.
\end{itemize}

\section{Screenshots (Platzhalter)}
Fügen Sie hier aktuelle Screenshots ein, sobald verfügbar.
\begin{figure}[H]
  \centering
  \fbox{\parbox[c][5cm][c]{12cm}{\centering Login-Screen (Platzhalter)}}
  \caption{Login und Rollenwahl}
\end{figure}

\begin{figure}[H]
  \centering
  \fbox{\parbox[c][5cm][c]{12cm}{\centering Klassenübersicht mit Fortschritt (Platzhalter)}}
  \caption{Klassenübersicht im Lehrkräftebereich}
\end{figure}

\begin{figure}[H]
  \centering
  \fbox{\parbox[c][5cm][c]{12cm}{\centering PDF-Vorschau und Export (Platzhalter)}}
  \caption{PDF-Export aus einer Lernentwicklungsansicht}
\end{figure}

\begin{figure}[H]
  \centering
  \fbox{\parbox[c][5cm][c]{12cm}{\centering Admin-Einstellungen (Platzhalter)}}
  \caption{Globale Einstellungen mit Branding und KI-Konfiguration}
\end{figure}

\begin{figure}[H]
  \centering
  \fbox{\parbox[c][5cm][c]{12cm}{\centering Schüler-QR-Login und TTS-Bar (Platzhalter)}}
  \caption{Schülerzugang mit Vorlesefunktion}
\end{figure}

\clearpage
\appendix
\section{Schnellanleitung für Lehrkräfte}
\textbf{Ziel:} Innerhalb weniger Minuten Berichte pro Klasse pflegen.

\begin{enumerate}
  \item Einloggen unter \url{.../teacher}.
  \item Klasse oder delegierten Fachbereich öffnen.
  \item Schüler auswählen \rightarrow Pflichtfelder und Freitextfelder ausfüllen.
  \item Fortschritt prüfen (Anzeige fehlender Felder beachten).
  \item Optional: KI-Textvorschläge einblenden und bei Bedarf übernehmen.
  \item Live-Vorschau kontrollieren und PDF exportieren (einzeln oder Klassenexport, falls freigeschaltet).
  \item Delegationen verwalten: Fachbereiche an Kolleg:innen delegieren oder zurückholen.
\end{enumerate}

\vspace{1cm}
{\small Stand: automatisch aus aktuellem Code abgeleitet. Bitte Anpassungen nach Software-Updates prüfen.}

\end{document}
