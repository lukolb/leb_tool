\documentclass[12pt,a4paper]{article}
\usepackage[utf8]{inputenc}
\usepackage[T1]{fontenc}
\usepackage[ngerman]{babel}
\usepackage{graphicx}
% --- Optional title icon (avoid compile errors if asset is missing) ---
\newcommand{\LEBTitleIcon}{%
  \IfFileExists{../assets/icons/lebtool-icon-512x512.png}{%
    \includegraphics[width=4cm]{../assets/icons/lebtool-icon-512x512.png}%
  }{%
    \fbox{\parbox[c][4cm][c]{4cm}{\centering\textbf{LEB Tool}}}%
  }%
}

\usepackage{geometry}
\usepackage{hyperref}
\usepackage{booktabs}
\usepackage{longtable}
\usepackage{float}
\usepackage{xcolor}
\usepackage{enumitem}
\usepackage{array}
\usepackage{tabularx}
\usepackage{csquotes}

\geometry{margin=2.5cm}
\setlength{\parskip}{6pt}
\setlength{\parindent}{0pt}

\definecolor{primary}{HTML}{0B57D0}
\definecolor{secondary}{HTML}{111111}
\hypersetup{
  colorlinks=true,
  linkcolor=primary,
  urlcolor=primary,
  citecolor=primary
}

\begin{document}

% Titelblatt
\pagenumbering{gobble}
\begin{center}
  \LEBTitleIcon\\[1cm]
  {\Huge \textbf{LEB Tool Benutzerhandbuch}}\\[0.2cm]
  {\Large Lernentwicklungsberichte digital verwalten}\\[1cm]
  {\large Stand: \today}\\[1.5cm]
\end{center}

\vfill
\begin{center}
  \textcolor{secondary}{\small Dieses Dokument beschreibt Funktionen und Bedienung des LEB Tools für Admins, Lehrkräfte, Schüler:innen und Elternzugänge.}
\end{center}
\clearpage

\pagenumbering{arabic}
\tableofcontents
\clearpage

\section{Überblick}
Das \textbf{LEB Tool} unterstützt Schulen bei der strukturierten Erstellung von Lernentwicklungsberichten:
\begin{itemize}
  \item Admin-Bereich zur Verwaltung von Nutzer:innen, Klassen, Templates, Feldern und Einstellungen.
  \item Lehrer-Bereich zur Bearbeitung von Berichten pro Klasse, inklusive Delegationen (Fachbereiche).
  \item Schüler-Bereich zur Selbsteinschätzung (sofern aktiviert), optional mit Vorlesefunktion (TTS).
  \item Eltern-Portal-Link zum Lesen des fertigen Berichts (und ggf. Feedback).
\end{itemize}

Die \textbf{führende Instanz} ist immer die PDF-Vorlage. Alle Felder, Gruppen und Exporte leiten sich daraus ab.\newline
\textbf{Zentraler Arbeitskontext} ist die Klasse: Nutzer:innen greifen immer über eine Klasse auf Berichte zu, nicht direkt über einzelne Personen.

\section{Rollen und Zugänge}
\subsection{Admin}
Admins verwalten das System: Nutzer:innen, Klassen, Templates, Felder, Einstellungen, Audit-Log und Exporte.
\begin{itemize}
  \item Installation/Initialkonfiguration inkl. Template-Upload.
  \item Verwaltung der fachlich zwingenden Einrichtungsreihenfolge (siehe Abschnitt \ref{sec:setup-sequenz}).
  \item Pflege stabiler Optionslisten-Values (keine nachträglichen Änderungen bestehender Werte).
  \item Klassen- und Nutzerzuordnung steuert, wer welche Berichte sieht.
\end{itemize}

\subsection{Lehrkraft}
Lehrkräfte bearbeiten Berichte für ihre Klassen. Zusätzlich können Fachbereiche an Kolleg:innen delegiert werden.

\subsection{Schüler:in}
Schüler:innen füllen (je nach Konfiguration) Teile des Berichts selbst aus. Optional steht eine Vorlesefunktion zur Verfügung.

\subsection{Eltern}
Eltern erhalten einen Link zum Elternportal, um den Bericht einzusehen. Optional kann ein Feedback-Formular verfügbar sein.

\section{Navigation und Layout}
Die Oberfläche nutzt ein einheitliches Layout mit Karten (\texttt{card}), Buttons (\texttt{btn}) und Tabellenstilen. Menüs können Untermenüs enthalten.
\begin{itemize}
  \item Pflichtfelder sind farblich oder mit Symbol gekennzeichnet.
  \item Fortschrittsanzeigen geben Rückmeldung zur Vollständigkeit pro Klasse/Schüler:in.
  \item Formularfelder sind nach Gruppen (Fächer/Verhaltensbereiche) gegliedert, analog zur Template-Struktur.
\end{itemize}

\section{Admin-Bereich}
\subsection{Einrichtungsreihenfolge}\label{sec:setup-sequenz}
Die fachlich zwingende Reihenfolge muss eingehalten werden, damit alle abhängigen Daten korrekt verknüpft sind:
\begin{enumerate}
  \item PDF-Template hochladen.
  \item Formularfelder extrahieren (automatisch aus der PDF-Vorlage).
  \item Felder gruppieren und sortieren (z.\,B. Deutsch, Mathematik, Sozialverhalten).
  \item Optionslisten erstellen (Values sind stabil, Labels rein visuell).
  \item Optionslisten den Feldern zuordnen (Dropdowns erhalten damit Werte).
  \item Klassen anlegen und Templates zuweisen (Klasse ist zentraler Kontext).
  \item Lehrkräfte und Schüler:innen anlegen und den Klassen zuordnen.
\end{enumerate}
Diese Schritte dürfen nicht ausgelassen oder umsortiert werden, da sonst Felder oder Exporte unvollständig bleiben.

\subsection{Dashboard}
Das Dashboard zeigt Systemstatus, ggf. Bearbeitungsstand und Quicklinks.

\subsection{Nutzerverwaltung}
Admins können Lehrkräfte und Admins anlegen, deaktivieren und Passwörter zurücksetzen.

\subsection{Klassenverwaltung}
Klassen enthalten u.\,a. Schuljahr, Stufe, Bezeichnung, Name, Template-Zuordnung und Aktiv-Status.
\begin{itemize}
  \item Mehrere Lehrkräfte können einer Klasse zugeordnet sein.
  \item Schüler:innen werden der Klasse zugeordnet und erhalten nur dadurch Zugriff auf eigene Felder.
  \item Archivierung deaktiviert die Bearbeitung, Exporte bleiben i.\,d.\,R. möglich (prüfen Sie lokale Rechte-Settings).
\end{itemize}

\subsection{Templates und Felder}
Templates definieren die Struktur eines Berichts. Felder besitzen u.\,a.:
\begin{itemize}
  \item Feldtyp (z.\,B. Text, Auswahl, Skala, Unterschrift)
  \item Pflichtfeld-Flag
  \item Reihenfolge/Sortierung
  \item Sprachlabels (mehrsprachig, falls aktiviert)
  \item Seitenzuordnung laut PDF
  \item Optionale Optionsliste (für Dropdown-Felder)
\end{itemize}
Hinweise:
\begin{itemize}
  \item Feldnamen können mehrfach vorkommen (z.\,B. identische Felder auf mehreren Seiten).
  \item Gruppen steuern die UI-Gliederung und Exportstruktur.
  \item Änderungen am Template wirken auf alle zugeordneten Klassen; prüfen Sie Auswirkungen vor dem Speichern.
\end{itemize}

\subsection{Einstellungen}
Systemweite Einstellungen steuern Features wie:
\begin{itemize}
  \item Sichtbarkeit/Verfügbarkeit von Sprachen
  \item Aktivierung von KI-Funktionen (sofern vorhanden)
  \item Elternportal-Optionen (Feedback anzeigen, Unterschriftenfeld, etc.)
  \item Export-Regeln und Rollenrechte
\end{itemize}

\subsection{Audit-Log}
Das Audit-Log protokolliert wichtige Aktionen. Filter, Sortierung und optionale Anzeige von IP-Adressen helfen bei der Analyse.

\subsection{Optionslisten}
Optionslisten stellen Dropdown-Werte bereit:
\begin{itemize}
  \item Jeder Eintrag besteht aus technischem \emph{Value} (stabil, unveränderbar) und sichtbarem \emph{Label} (änderbar).
  \item Werden Werte inhaltlich geändert, muss eine neue Optionsliste erstellt oder versioniert werden.
  \item Leere Dropdowns deuten oft auf fehlende oder falsch zugeordnete Optionslisten hin.
\end{itemize}

\section{Lehrer-Bereich}
\subsection{Klassenübersicht}
Lehrkräfte sehen ihre aktiven Klassen und können Klassen archivieren (deaktivieren). Pro Klasse wird ggf. ein Bearbeitungsstand angezeigt.

\subsection{Bericht bearbeiten}
In der Eingabemaske werden Felder gemäß Template angezeigt:
\begin{itemize}
  \item Pflichtfelder sind markiert.
  \item Fortschritt zeigt fehlende Felder.
  \item Tages-/Abschnittsstruktur nach Template.
  \item Gruppierte Ansicht (z.\,B. Fächer/Verhalten) erleichtert Fach-zu-Fach-Bearbeitung.
\end{itemize}
Empfohlener Ablauf:
\begin{enumerate}
  \item Klasse öffnen und klassenweite Felder zuerst ausfüllen.
  \item Pro Fach/Gruppe arbeiten, nicht Kind für Kind.
  \item Zwischenspeichern und offenen Pflichtfelder-Status prüfen.
  \item Bei Bedarf Delegation nutzen (siehe unten) oder Schülerfreigabe aktivieren.
\end{enumerate}

\subsection{Delegationen}
Fachbereiche können an andere Lehrkräfte delegiert werden. Delegierte sehen diese in ihrer Delegations-Inbox.
\begin{itemize}
  \item Delegationsstatus kann angepasst werden.
  \item Delegationen können neu zugewiesen oder entfernt werden.
  \item Statuspfad: offen \(\rightarrow\) in Bearbeitung \(\rightarrow\) abgeschlossen (zurücksetzbar)
\end{itemize}

\subsection{Schülerfreigabe}
\begin{itemize}
  \item Standard: Schüler-Eingabe ist gesperrt.
  \item Freigabe wird aktiv gesetzt und gilt nur im definierten Zeitfenster.
  \item Lehrkräfte können Freigaben jederzeit schließen; danach ist kein Schüler-Schreibzugriff mehr möglich.
  \item Schüler:innen sehen nur eigene Daten und nur freigegebene Felder.
\end{itemize}

\section{Schüler-Bereich}
\subsection{Login}
Schüler:innen loggen sich je nach Setup ein (z.\,B. über QR-Code oder Zugangsdaten).

\subsection{Eingabe}
Schüler:innen füllen zugewiesene Felder aus. Pflichtfelder können erzwungen werden.

\subsection{Vorlesefunktion (TTS)}
Falls aktiviert, kann Text vorgelesen werden. Für eine gute Nutzererfahrung sind passende Stimmen und Browser-Kompatibilität wichtig.

\section{Elternportal}
\subsection{Zugriff}
Eltern erhalten einen Link (z.\,B. per E-Mail). Der Link führt zu einer Ansicht des Berichts (PDF oder HTML) und optional zu einem Feedback-Formular.

\subsection{Unterschriftenfeld}
Falls das Template ein Unterschriftenfeld enthält, kann der Name der anfragenden Lehrkraft automatisch eingetragen werden.

\section{Export}
\subsection{Export-API}
Exporte werden über eine gemeinsame API-Schicht umgesetzt. Zugriffsrechte hängen von Rolle und Klassenberechtigung ab.

\subsection{PDF-Export}
Beim PDF-Export wird auf korrekte Datumsformatierung geachtet (z.\,B. \texttt{01.12.2026}), ebenso auf Zeichencodierung und Layout.
\begin{itemize}
  \item Lehrkräfte: Einzel-Schüler-Export (Test) oder Klassenexport.
  \item Admin: Klassenübergreifender Export (Filter nach Schuljahr oder Klasse).
  \item Export basiert immer auf Template + gespeicherten Daten; die PDF ist führend.
\end{itemize}

\appendix
\section{Platzhalter-Screenshots}
\begin{figure}[H]
  \centering
  \fbox{\parbox[c][5cm][c]{12cm}{\centering Admin-Dashboard (Platzhalter)}}
  \caption{Admin-Dashboard Übersicht}
\end{figure}

\begin{figure}[H]
  \centering
  \fbox{\parbox[c][5cm][c]{12cm}{\centering Lehrer-Eingabemaske (Platzhalter)}}
  \caption{Lehrer: Bericht bearbeiten}
\end{figure}

\begin{figure}[H]
  \centering
  \fbox{\parbox[c][5cm][c]{12cm}{\centering Delegations-Inbox (Platzhalter)}}
  \caption{Delegations-Inbox}
\end{figure}

\begin{figure}[H]
  \centering
  \fbox{\parbox[c][5cm][c]{12cm}{\centering Schüler-QR-Login und TTS-Bar (Platzhalter)}}
  \caption{Schülerzugang mit Vorlesefunktion}
\end{figure}

\clearpage
\section{Schnellanleitung für Lehrkräfte}
\textbf{Ziel:} Innerhalb weniger Minuten Berichte pro Klasse pflegen.

\begin{enumerate}
  \item Einloggen unter \textbf{\url{TODO: Lehrer-URL einfügen}}.
  \item Klasse oder delegierten Fachbereich öffnen.
  \item Schüler auswählen \(\rightarrow\) Pflichtfelder und Freitextfelder ausfüllen.
  \item Fortschritt prüfen (Anzeige fehlender Felder beachten).
  \item Bei Bedarf delegieren: Fachbereiche an Kolleg:innen abgeben.
  \item Delegationen verwalten: Fachbereiche an Kolleg:innen delegieren oder zurückholen.
  \item Optional Schülerfreigabe setzen und nach Ablauf wieder schließen.
  \item Export je Schüler oder gesamte Klasse durchführen (PDF prüfen).
\end{enumerate}

\vspace{1cm}
{\small Stand: automatisch aus aktuellem Code abgeleitet. Bitte Anpassungen nach Software-Updates prüfen.}

\end{document}