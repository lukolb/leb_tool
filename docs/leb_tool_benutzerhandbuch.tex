\documentclass[12pt,a4paper]{article}
\usepackage[utf8]{inputenc}
\usepackage[T1]{fontenc}
\usepackage[ngerman]{babel}
\usepackage{graphicx}
% --- Optional title icon (avoid compile errors if asset is missing) ---
\newcommand{\LEBTitleIcon}{%
  \IfFileExists{../assets/icons/lebtool-icon-512x512.png}{%
    \includegraphics[width=4cm]{../assets/icons/lebtool-icon-512x512.png}%
  }{%
    \fbox{\parbox[c][4cm][c]{4cm}{\centering\textbf{LEB Tool}}}%
  }%
}

\usepackage{geometry}
\usepackage{hyperref}
\usepackage{booktabs}
\usepackage{longtable}
\usepackage{float}
\usepackage{xcolor}
\usepackage{enumitem}
\usepackage{array}
\usepackage{tabularx}
\usepackage{csquotes}
\usepackage{fancyhdr} % Kopf- und Fußzeilen

\geometry{margin=2.5cm}
\setlength{\parskip}{6pt}
\setlength{\parindent}{0pt}

\definecolor{primary}{HTML}{0B57D0}
\definecolor{secondary}{HTML}{111111}
\hypersetup{
  colorlinks=true,
  linkcolor=primary,
  urlcolor=primary,
  citecolor=primary
}

\pagestyle{fancy}
\fancyhf{}
\lhead{LEB Tool Benutzerhandbuch}
\chead{Stand: \today}
\rhead{PDF als führende Instanz}
\cfoot{\thepage}

\setlist[itemize]{noitemsep, topsep=2pt}
\setlist[enumerate]{noitemsep, topsep=2pt}
\setcounter{tocdepth}{2}

\begin{document}

% Titelblatt
\pagenumbering{gobble}
\begin{center}
  \LEBTitleIcon\\[1cm]
  {\Huge \textbf{LEB Tool Benutzerhandbuch}}\\[0.2cm]
  {\Large Lernentwicklungsberichte digital verwalten}\\[1cm]
  {\large Stand: \today}\\[1.5cm]
\end{center}

\vfill
\begin{center}
  \textcolor{secondary}{\small Dieses Dokument beschreibt Funktionen und Bedienung des LEB Tools für Admins, Lehrkräfte, Schüler:innen und Elternzugänge.}
\end{center}
\clearpage

\pagenumbering{arabic}
\tableofcontents
\clearpage

\section{Überblick: Was ist das LEB Tool?}
Das \textbf{LEB Tool} unterstützt Schulen bei der strukturierten Erstellung von Lernentwicklungsberichten (LEB). Kernidee: Eine PDF-Vorlage mit Formularfeldern ist die führende Instanz. Das Tool extrahiert diese Felder, strukturiert sie in Gruppen, verknüpft sie bei Bedarf mit Optionslisten und erzeugt wieder eine ausgefüllte, druckfertige PDF. \textbf{Zentraler Arbeitskontext} ist die Klasse: Zugriffe erfolgen stets über die Klasse, nicht direkt über einzelne Personen.

\textbf{Abb. 1 – Login / Rollen-Einstieg}\
Platzhalter für Screenshot der Start- bzw. Login-Seite mit erkennbaren Zugängen für Admin, Lehrkraft und Schüler:in.

\subsection{Rollen \& Rechte}
\begin{tabularx}{\textwidth}{|l|X|l|}
\hline
\textbf{Rolle} & \textbf{Aufgaben} & \textbf{Typisch wann?}\\
\hline
Admin & Installation, Templates, Optionslisten, Klassen/Benutzer, Gesamt-Export & Setup/Änderungen\\
\hline
Lehrkraft & Klassenfelder, Schülerberichte, Freigaben, Export & Arbeitsphase\\
\hline
Schüler:in & Eigene Reflexionsfelder (nur nach Freigabe) & Kurzzeit-Fenster\\
\hline
\end{tabularx}

\noindent\fbox{\parbox{\textwidth}{\textbf{Hinweis:} Schüler:innen haben niemals dauerhaften Schreibzugriff. Die Freigabe ist immer zeitlich begrenzt und wird bewusst von der Lehrkraft gesteuert.}}

\section{Installation \& Erstinbetriebnahme (Admin)}
Die Installation erfolgt einmalig. Anschließend folgt die fachliche Einrichtung mit Vorlage, Feldern, Optionslisten und Klassen.

\subsection{Voraussetzungen}
\begin{itemize}
  \item Webserver mit PHP.
  \item MySQL/MariaDB-Datenbank.
  \item FTP-Zugang zum Webspace.
  \item HTTPS empfohlen (Datenschutz).
\end{itemize}

\subsection{Installation Schritt für Schritt}
\begin{enumerate}
  \item Dateien hochladen: Projektordner per FTP in das Zielverzeichnis kopieren.
  \item \texttt{install.php} aufrufen: Im Browser \texttt{.../install.php} öffnen.
  \item Datenbank verbinden: Host, DB-Name, Nutzer, Passwort eintragen.
  \item Admin anlegen: Admin-Login erstellen (starkes Passwort).
  \item Login testen: Direkt danach im Adminbereich anmelden.
  \item Sicherheit: \texttt{install.php} entfernen oder per Server-Regel sperren (sofern vorgesehen).
\end{enumerate}

\textbf{Abb. 2 – Installationsmaske}\
Platzhalter für Screenshot der Seite \texttt{install.php} mit Datenbankfeldern und Abschluss-Button.

\noindent\fbox{\parbox{\textwidth}{\textbf{Warnung:} Falsche Datenbank-Zugangsdaten oder fehlende PHP-Erweiterungen führen zu Installationsfehlern. Prüfen Sie Host, Port und Rechte.}}

\section{Admin: Einrichtung in der richtigen Reihenfolge}\label{sec:setup-sequenz}
Die fachlich zwingende Reihenfolge muss eingehalten werden, damit alle abhängigen Daten korrekt verknüpft sind.

\subsection{Vorlage (PDF) hochladen}
\begin{enumerate}
  \item Adminbereich \(\rightarrow\) Templates öffnen.
  \item PDF hochladen und benennen.
  \item Vorlage speichern und aktivieren.
\end{enumerate}

\textbf{Abb. 3 – Template-Upload}\
Platzhalter für Upload-Maske: PDF auswählen, Name vergeben, Speichern/Aktivieren.

\subsection{Felder extrahieren (Formularfelder auslesen)}
\begin{enumerate}
  \item Aktion \enquote{Felder extrahieren} starten.
  \item Anzahl prüfen: passt sie zur Vorlage?
  \item Bei wiederholten Feldnamen sicherstellen, dass alle Instanzen gefunden werden.
  \item Seitenzuordnung stichprobenartig prüfen (Seite 1 Felder von Seite 1).
\end{enumerate}

\textbf{Abb. 4 – Feldliste nach Extraktion}\
Platzhalter für Liste der Feldnamen inkl. Seiten und Status.

\subsection{Felder gruppieren \& Reihenfolge festlegen}
\begin{enumerate}
  \item Hauptgruppen anlegen (z.\,B. Deutsch, Mathematik, Sozialverhalten).
  \item Felder per Auswahl/Batch zuordnen.
  \item Filter nutzen, um Varianten sauber zu trennen.
  \item Reihenfolge speichern und in der Lehreransicht testen.
\end{enumerate}

\textbf{Abb. 5 – Gruppierung \& Filter}\
Platzhalter für Gruppenliste mit Batch-Zuordnung und Filtern.

\subsection{Optionslisten erstellen (Noten/Skalen)}
\begin{enumerate}
  \item Admin \(\rightarrow\) Optionslisten \(\rightarrow\) Neue Liste.
  \item Name eindeutig wählen (z.\,B. \enquote{Skala 1--5}).
  \item Einträge anlegen: technischer Value (stabil) + Label (Anzeige).
  \item Speichern und testen.
\end{enumerate}

\noindent\fbox{\parbox{\textwidth}{\textbf{Hinweis:} Labels dürfen angepasst werden. Values bleiben stabil. Bei inhaltlichen Änderungen bitte eine neue Optionsliste anlegen oder versionieren.}}

\textbf{Abb. 6 – Optionslisten-Editor}\
Platzhalter für Ansicht zum Bearbeiten einer Optionsliste mit Einträgen.

\subsection{Optionslisten den Feldern zuordnen}
\begin{enumerate}
  \item Feld öffnen und Feldtyp Option/Note/Skala setzen.
  \item Passende Optionsliste auswählen.
  \item Batch-Zuordnung nutzen, wenn mehrere Felder identisch sind.
\end{enumerate}

\textbf{Abb. 7 – Zuordnung Optionsliste \(\rightarrow\) Feld}\
Platzhalter für Feldbearbeitung mit Auswahl einer Optionsliste.

\subsection{Klassen anlegen \& Template zuweisen}
\begin{enumerate}
  \item Admin \(\rightarrow\) Klassen \(\rightarrow\) Neu.
  \item Schuljahr, Stufe, Bezeichnung (z.\,B. 4a) eintragen.
  \item Template auswählen und Klasse aktiv setzen.
\end{enumerate}

\textbf{Abb. 8 – Klasse anlegen}\
Platzhalter für Klassenformular mit Template-Auswahl.

\subsection{Lehrkräfte \& Schüler:innen verwalten}
\textbf{Lehrkräfte}
\begin{itemize}
  \item Lehrkraft anlegen (Name, Login, Passwort).
  \item Lehrkraft den Klassen zuordnen.
\end{itemize}

\textbf{Schüler:innen}
\begin{itemize}
  \item Schüler:innen anlegen oder importieren (Vorname, Nachname, Klasse).
  \item Schüler-Login vergeben oder erzeugen; Login-Liste für die Klasse ausgeben.
\end{itemize}

\textbf{Abb. 9 – Benutzer \& Zuordnungen}\
Platzhalter für Ansicht zum Anlegen und Zuweisen von Lehrkräften und Schüler:innen.

\section{Lehrkräfte: Berichte erstellen}
Einstieg für Lehrkräfte: Klasse wählen, klassenweite Felder ausfüllen, Schülerberichte ergänzen, Freigabe steuern, Export durchführen.

\subsection{Klasse auswählen \& Überblick}
\begin{enumerate}
  \item Anmelden im Lehrerbereich.
  \item Klasse wählen und Status (Freigabe, Bearbeitungsstand, Export) prüfen.
\end{enumerate}

\textbf{Abb. 10 – Lehrerbereich \textendash{} Übersicht}\
Platzhalter für Klassenliste/Startansicht mit Status.

\subsection{Klassenweite Felder (einmal \textendash{} für alle)}
\begin{enumerate}
  \item Bereich \enquote{Klassenfelder} öffnen.
  \item Felder vollständig ausfüllen und speichern.
\end{enumerate}

\textbf{Abb. 11 – Lehrerbereich \textendash{} Klassenfelder}\
Platzhalter für Ansicht der klassenweiten Felder.

\subsection{Schülerberichte (individuell)}
\begin{enumerate}
  \item Schüler auswählen.
  \item Fach/Gruppe wählen (z.\,B. Deutsch).
  \item Textfelder ausfüllen, Optionsfelder setzen, speichern.
\end{enumerate}

\noindent\fbox{\parbox{\textwidth}{\textbf{Hinweis:} Zeit sparen durch Arbeiten \enquote{Fach für Fach}: pro Gruppe nacheinander alle Schüler:innen ausfüllen, statt pro Kind den gesamten Bericht.}}

\textbf{Abb. 12 – Lehrerbereich \textendash{} Schülerbericht}\
Platzhalter für Einzelansicht eines Schülers mit Gruppen und Eingabefeldern.

\subsection{Schüler-Eingabe freigeben (kontrolliert)}
\begin{enumerate}
  \item Freigabe aktivieren und Zeitfenster festlegen.
  \item Hinweistext für Schüler:innen hinterlegen (kurz, klar).
  \item Nach Ablauf Freigabe schließen; danach kein Schüler-Schreibzugriff.
\end{enumerate}

\textbf{Abb. 13 – Lehrerbereich \textendash{} Freigabe}\
Platzhalter für Freigabe-Schalter und Hinweistext.

\section{Schüler:innen: Eingaben machen}
Schüler:innen können nur schreiben, wenn die Lehrkraft die Eingabe freigegeben hat. Standardzustand ist gesperrt.

\subsection{Gesperrt vs. Freigegeben}
\textbf{Abb. 14 – Schülerbereich \textendash{} gesperrt}\
Platzhalter für Seite mit Hinweis: Eingabe noch nicht freigegeben.

\textbf{Abb. 15 – Schülerbereich \textendash{} freigegeben}\
Platzhalter für Eingabemaske mit Feldern und Speichern.

\noindent\fbox{\parbox{\textwidth}{\textbf{Hinweis für Schüler:innen (Copy \& Paste):} Schreibe 3--5 Sätze, nenne mindestens ein Beispiel, bleib freundlich und ehrlich, speichere am Ende.}}

\section{PDF-Export \& Archivierung}
Alle Exporte basieren auf Template + gespeicherten Daten; die PDF bleibt führend.

\subsection{Export durch Lehrkräfte}
\begin{enumerate}
  \item Exportbereich öffnen.
  \item Test: einen Schüler exportieren und prüfen.
  \item Klassenexport durchführen und PDF stichprobenartig kontrollieren.
\end{enumerate}

\textbf{Abb. 16 – Export im Lehrerbereich}\
Platzhalter für Export-Buttons/Optionen im Lehrerbereich.

\subsection{Export durch Admin (alle Klassen)}
\begin{enumerate}
  \item Admin \(\rightarrow\) Export/Reports öffnen.
  \item Filter setzen (z.\,B. Schuljahr oder Klasse).
  \item Gesamtexport starten.
\end{enumerate}

\textbf{Abb. 17 – Export im Adminbereich}\
Platzhalter für Filter + Start-Export im Adminbereich.

\section{FAQ \& Troubleshooting}
\begin{longtable}{|p{0.45\textwidth}|p{0.45\textwidth}|}
\hline
\textbf{Problem} & \textbf{Lösung}\\
\hline
Dropdown zeigt keine Werte & Optionsliste fehlt oder wurde dem Feld nicht zugeordnet; Liste prüfen und zuweisen.\\
\hline
Schüler:innen sehen keine Eingabe & Freigabe durch Lehrkraft fehlt oder wurde geschlossen; Freigabe erneut aktivieren.\\
\hline
Felder erscheinen auf falschen Seiten & Felder erneut extrahieren und Seitenzuordnung prüfen.\\
\hline
Nach Optionslisten-Änderung stimmen Werte nicht & Values wurden geändert; neue Optionsliste anlegen oder Version wählen, Values stabil halten.\\
\hline
Admin-Export leer & Prüfen, ob Klassen Templates zugeordnet sind und ob die Exportrolle korrekt ist.\\
\hline
\end{longtable}

\noindent\fbox{\parbox{\textwidth}{\textbf{Checkliste vor dem Zeugnisdruck:} Klassenfelder vollständig, zwei Schülerberichte stichprobenartig geprüft, Schüler-Freigabe geschlossen, PDF-Export (ein Schüler + Klasse) getestet, Ablage-/Archivordner vorbereitet.}}

\appendix
\section{Platzhalter-Screenshots}
\begin{figure}[H]
  \centering
  \fbox{\parbox[c][5cm][c]{12cm}{\centering Admin-Dashboard (Platzhalter)}}
  \caption{Admin-Dashboard Übersicht}
\end{figure}

\begin{figure}[H]
  \centering
  \fbox{\parbox[c][5cm][c]{12cm}{\centering Lehrer-Eingabemaske (Platzhalter)}}
  \caption{Lehrer: Bericht bearbeiten}
\end{figure}

\begin{figure}[H]
  \centering
  \fbox{\parbox[c][5cm][c]{12cm}{\centering Delegations-Inbox (Platzhalter)}}
  \caption{Delegations-Inbox}
\end{figure}

\begin{figure}[H]
  \centering
  \fbox{\parbox[c][5cm][c]{12cm}{\centering Schüler-QR-Login und TTS-Bar (Platzhalter)}}
  \caption{Schülerzugang mit Vorlesefunktion}
\end{figure}

\clearpage
\section{Schnellanleitung für Lehrkräfte}
\textbf{Ziel:} Innerhalb weniger Minuten Berichte pro Klasse pflegen.

\begin{enumerate}
  \item Einloggen unter \textbf{\url{TODO: Lehrer-URL einfügen}}.
  \item Klasse oder delegierten Fachbereich öffnen.
  \item Schüler auswählen \(\rightarrow\) Pflichtfelder und Freitextfelder ausfüllen.
  \item Fortschritt prüfen (Anzeige fehlender Felder beachten).
  \item Bei Bedarf delegieren: Fachbereiche an Kolleg:innen abgeben.
  \item Delegationen verwalten: Fachbereiche an Kolleg:innen delegieren oder zurückholen.
  \item Optional Schülerfreigabe setzen und nach Ablauf wieder schließen.
  \item Export je Schüler oder gesamte Klasse durchführen (PDF prüfen).
\end{enumerate}

\vspace{1cm}
{\small Stand: automatisch aus aktuellem Code abgeleitet. Bitte Anpassungen nach Software-Updates prüfen.}

\end{document}
