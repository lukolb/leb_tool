\documentclass[12pt,a4paper]{article}
\usepackage[utf8]{inputenc}
\usepackage[T1]{fontenc}
\usepackage[ngerman]{babel}
\usepackage{graphicx}
% --- Optional title icon (avoid compile errors if asset is missing) ---
\newcommand{\LEBTitleIcon}{%
  \IfFileExists{../assets/icons/lebtool-icon-512x512.png}{%
    \includegraphics[width=4cm]{../assets/icons/lebtool-icon-512x512.png}%
  }{%
    \fbox{\parbox[c][4cm][c]{4cm}{\centering\textbf{LEB Tool}}}%
  }%
}

\usepackage{geometry}
\usepackage{hyperref}
\usepackage{booktabs}
\usepackage{longtable}
\usepackage{float}
\usepackage{xcolor}
\usepackage{enumitem}
\usepackage{array}
\usepackage{tabularx}
\usepackage{csquotes}
\usepackage{fancyhdr} % Kopf- und Fußzeilen
\usepackage{makeidx}  % Indexerstellung

\geometry{margin=2.5cm}
\setlength{\parskip}{6pt}
\setlength{\parindent}{0pt}

\definecolor{primary}{HTML}{0B57D0}
\definecolor{secondary}{HTML}{111111}
\definecolor{hintbg}{HTML}{EEF3FF}
\definecolor{warnbg}{HTML}{FFF4E5}
\definecolor{warnborder}{HTML}{E29500}
\hypersetup{
  colorlinks=true,
  linkcolor=primary,
  urlcolor=primary,
  citecolor=primary
}

\newcommand{\HandbuchStand}{25.12.2025}

\newcommand{\HintBox}[1]{%
  \noindent\fcolorbox{primary}{hintbg}{\parbox{\dimexpr\linewidth-2\fboxsep-2\fboxrule\relax}{\textbf{Hinweis:} #1}}\par
}

\newcommand{\WarnBox}[1]{%
  \noindent\fcolorbox{warnborder}{warnbg}{\parbox{\dimexpr\linewidth-2\fboxsep-2\fboxrule\relax}{\textbf{Warnung:} #1}}\par
}

\newcommand{\PlaceholderFigure}[2]{%
  \begin{figure}[H]
    \centering
    \fcolorbox{primary}{white}{\parbox[c][5cm][c]{12cm}{\centering #2}}
    \caption{#1}
  \end{figure}
}

\pagestyle{fancy}
\fancyhf{}
\lhead{LEB Tool \textendash{} Benutzerhandbuch}
\rhead{}
\lfoot{Stand: \HandbuchStand}
\rfoot{\thepage}
\renewcommand{\headrulewidth}{0.4pt}
\renewcommand{\footrulewidth}{0.4pt}

\fancypagestyle{plain}{%
  \fancyhf{}
  \lhead{LEB Tool \textendash{} Benutzerhandbuch}
  \rhead{}
  \lfoot{Stand: \HandbuchStand}
  \rfoot{\thepage}
  \renewcommand{\headrulewidth}{0.4pt}
  \renewcommand{\footrulewidth}{0.4pt}
}

\setlist[itemize]{noitemsep, topsep=2pt}
\setlist[enumerate]{noitemsep, topsep=2pt}
\setcounter{tocdepth}{2}

\makeindex

\begin{document}

% Titelblatt
\pagenumbering{gobble}
\thispagestyle{empty}
\begin{center}
  \LEBTitleIcon\\[1cm]
  {\Huge \textbf{LEB Tool \textendash{} Benutzerhandbuch}}\\[0.3cm]
  {\Large Schritt-für-Schritt Anleitung für Admins, Lehrkräfte und Schüler:innen}\\[0.5cm]
  \textcolor{primary}{\rule{0.72\textwidth}{1.2pt}}\\[0.4cm]
  {\large Lernentwicklungsberichte digital verwalten}\\[0.2cm]
  {\large Zielgruppe: Grundschule · Kollegium · Administration}\\[0.2cm]
  {\large Stand: \HandbuchStand}\\[0.2cm]
  {\large Format: Hybrid \textendash{} vollständiger Text + Screenshot-Platzhalter (Abb.)}\\[0.2cm]
  \textcolor{primary}{\rule{0.72\textwidth}{1.2pt}}\\[0.6cm]
  {\normalsize \textit{Die PDF ist führend · Die Klasse ist der zentrale Kontext · Freigaben sind zeitlich begrenzt}}\\[0.3cm]
  {\normalsize \textit{Values in Optionslisten bleiben stabil · Stabilität vor maximaler Flexibilität}}
\end{center}

\vfill
\begin{center}
  \textcolor{secondary}{\small Dieses Dokument beschreibt Funktionen und Bedienung des LEB Tools für Admins, Lehrkräfte, Schüler:innen und Elternzugänge.}
\end{center}
\clearpage

\pagenumbering{arabic}
\tableofcontents
\clearpage

\section{Geltungsbereich \& Versionierung}
Dieses Handbuch richtet sich an Admins, Lehrkräfte und (wo vorgesehen) Schüler:innen der Grundschule. Es beschreibt Bedienung und Abläufe des LEB Tools für die Erstellung von Lernentwicklungsberichten.
\begin{itemize}
  \item \textbf{Zweck:} Schritt-für-Schritt-Anleitung für Einrichtung, Nutzung und Export der Berichte.
  \item \textbf{Gültig für:} aktuelle Produktivversion zum Stand \HandbuchStand{}; Anpassungen nach Software-Updates sind einzuarbeiten.
  \item \textbf{Abdeckung:} Rollen, Einrichtungsreihenfolge, Freigaben, Export, typische Fehler, Screenshots als Platzhalter.
  \item \textbf{Nicht enthalten:} Quellcode-Details, juristische Datenschutzvorgaben, individuelle Schul-IT-Spezifika.
  \item \textbf{Versionierung:} Änderungen am Handbuch sollten mit Datum und Autor dokumentiert werden; ältere Fassungen archivieren.
\end{itemize}

\section{Überblick: Was ist das LEB Tool?}
Das \textbf{LEB Tool} unterstützt Schulen bei der strukturierten Erstellung von Lernentwicklungsberichten (LEB). Kernidee: Eine PDF-Vorlage mit Formularfeldern ist die führende Instanz. Das Tool extrahiert diese Felder, strukturiert sie in Gruppen, verknüpft sie bei Bedarf mit Optionslisten und erzeugt wieder eine ausgefüllte, druckfertige PDF. \textbf{Zentraler Arbeitskontext} ist die Klasse: Zugriffe erfolgen stets über die Klasse, nicht direkt über einzelne Personen.

\textbf{Kernobjekte und Prinzipien}
\begin{itemize}
  \item \textbf{Template (PDF):}\index{Template} definiert Feldnamen, Seiten und Struktur. Änderungen wirken auf alle zugeordneten Klassen.
  \item \textbf{Felder:} besitzen Typ (Text/Option), Gruppe, Reihenfolge, Seiteninfo und optional eine Optionsliste. Feldnamen können mehrfach vorkommen.\index{Feldextraktion|see{Felder}}
  \item \textbf{Optionslisten:}\index{Optionsliste} bestehen aus stabilen Values und änderbaren Labels; bei größeren Änderungen wird eine neue Liste versioniert.
  \item \textbf{Klasse:} Schuljahr, Stufe, Bezeichnung, Template, zugeordnete Lehrkräfte und Schüler:innen bilden den primären Arbeitskontext.\index{Klasse}
  \item \textbf{Elternansicht (optional):} Lesender Zugriff auf fertige Berichte kann bereitgestellt werden, ohne Schreibrechte zu vergeben.
  \item \textbf{Freigabeprinzip:}\index{Freigabe} Schüler-Eingabe ist standardmäßig gesperrt und wird nur zeitlich begrenzt geöffnet.
  \item \textbf{Exportlogik:}\index{Export} Berichte werden stets aus Template + gespeicherten Daten erzeugt; die PDF ist führend.
\end{itemize}

\HintBox{Best Practices: Klassenbezug immer zuerst prüfen, keine Values in bestehenden Optionslisten ändern, Freigaben bewusst setzen und wieder schließen.}

\PlaceholderFigure{Abb. 1 – Login / Rollen-Einstieg}{Platzhalter: Start- bzw. Login-Seite mit erkennbaren Zugängen für Admin, Lehrkraft und Schüler:in. Aufnahme: Browser öffnen und Tool-URL aufrufen.}

\subsection{Rollen \& Rechte}
\begin{tabularx}{\textwidth}{|l|X|l|}
\hline
\textbf{Rolle} & \textbf{Aufgaben} & \textbf{Typisch wann?}\\
\hline
Admin & Installation, Templates, Optionslisten, Klassen/Benutzer, Gesamt-Export & Setup/Änderungen\\
\hline
Lehrkraft & Klassenfelder, Schülerberichte, Freigaben, Export & Arbeitsphase\\
\hline
Schüler:in & Eigene Reflexionsfelder (nur nach Freigabe) & Kurzzeit-Fenster\\
\hline
\end{tabularx}

\HintBox{Schüler:innen haben niemals dauerhaften Schreibzugriff. Die Freigabe ist immer zeitlich begrenzt und wird bewusst von der Lehrkraft gesteuert.}

\section{Datenschutz \& Verantwortlichkeiten}
Kurzer Leitfaden (nicht rechtsverbindlich):
\begin{itemize}
  \item \textbf{Rollenbasierter Zugriff:} Admins verwalten Templates, Klassen und Nutzer; Lehrkräfte sehen nur ihre Klassen; Schüler:innen sehen ausschließlich eigene, freigegebene Felder.
  \item \textbf{Schüler-Logins:} Ausgabe der Zugangsdaten geschützt verwahren; bei Verlust Passwörter zurücksetzen.
  \item \textbf{Transport-Sicherheit:} Zugriff nach Möglichkeit per HTTPS; Testdaten vor Produktivstart nutzen, um echte Schülerdaten zu schützen.
  \item \textbf{Verantwortlichkeiten:} Schulen regeln intern, wer Datenschutzanfragen beantwortet (z.\,B. Admin, IT, Datenschutzbeauftragte:r). \textbf{TODO: Interne Ansprechperson und Kontaktweg ergänzen.}
  \item \textbf{Datenminimierung:} Nur notwendige Felder aktivieren; Exportablage an einem berechtigten Speicherort aufbewahren.
\end{itemize}

\WarnBox{Zugangsdaten nicht per unsichere Kanäle teilen.}

\section{Installation \& Erstinbetriebnahme (Admin)}
Die Installation erfolgt einmalig. Anschließend folgt die fachliche Einrichtung mit Vorlage, Feldern, Optionslisten und Klassen.

\subsection{Voraussetzungen}
\begin{itemize}
  \item Webserver mit PHP.
  \item MySQL/MariaDB-Datenbank.
  \item FTP-Zugang zum Webspace.
  \item HTTPS empfohlen (Datenschutz).
\end{itemize}

\subsection{Installation Schritt für Schritt}
\begin{enumerate}
  \item Dateien hochladen: Projektordner per FTP in das Zielverzeichnis kopieren.
  \item \texttt{install.php} aufrufen: Im Browser \texttt{.../install.php} öffnen.
  \item Datenbank verbinden: Host, DB-Name, Nutzer, Passwort eintragen.
  \item Admin anlegen: Admin-Login erstellen (starkes Passwort).
  \item Login testen: Direkt danach im Adminbereich anmelden.
  \item Sicherheit: \texttt{install.php} entfernen oder per Server-Regel sperren (sofern vorgesehen).
\end{enumerate}

\textbf{Nach der Installation}
\begin{itemize}
  \item Prüfen Sie Dateiberechtigungen für \texttt{storage}/Export-Ordner, damit PDF-Schreibrechte vorhanden sind.
  \item Legen Sie ein Backup der PDF-Template-Version ab, auf die sich die folgenden Schritte beziehen.
  \item Dokumentieren Sie den ersten Admin-Account (Name, E-Mail) für spätere Supportfälle.
\end{itemize}

\PlaceholderFigure{Abb. 2 – Installationsmaske}{Platzhalter: Seite \texttt{install.php} mit Datenbankfeldern und Abschluss-Button. Aufnahme: während der Installation vor Abschluss.}

\WarnBox{Falsche Datenbank-Zugangsdaten oder fehlende PHP-Erweiterungen führen zu Installationsfehlern. Prüfen Sie Host, Port und Rechte.}

\section{Admin: Einrichtung in der richtigen Reihenfolge}\label{sec:setup-sequenz}
Die fachlich zwingende Reihenfolge muss eingehalten werden, damit alle abhängigen Daten korrekt verknüpft sind. Jede Stufe baut auf der vorherigen auf: erst wenn Felder existieren, können sie gruppiert werden; erst wenn Gruppen und Optionslisten definiert sind, lohnt sich die Klassenanlage.

\subsection{Vorlage (PDF) hochladen}
\begin{enumerate}
  \item Adminbereich \(\rightarrow\) Templates öffnen.
  \item PDF hochladen und benennen.
  \item Vorlage speichern und aktivieren.
\end{enumerate}

\PlaceholderFigure{Abb. 3 – Template-Upload}{Platzhalter: Upload-Maske mit Dateiauswahl, Namensfeld und Speichern/Aktivieren-Buttons. Aufnahme: Admin → Templates.}

\subsection{Felder extrahieren (Formularfelder auslesen)}\index{Feldextraktion}
\begin{enumerate}
  \item Aktion \enquote{Felder extrahieren} starten.
  \item Anzahl prüfen: passt sie zur Vorlage?
  \item Bei wiederholten Feldnamen sicherstellen, dass alle Instanzen gefunden werden.
  \item Seitenzuordnung stichprobenartig prüfen (Seite 1 Felder von Seite 1).
\end{enumerate}

\HintBox{Vergleichen Sie die Feldanzahl mit der PDF: Abweichungen bedeuten oft, dass die Vorlage versteckte Felder oder nicht unterstützte Elemente enthält. Bei Dubletten hilft ein Export der Feldliste zum Abgleich.}

\PlaceholderFigure{Abb. 4 – Feldliste nach Extraktion}{Platzhalter: Liste der extrahierten Feldnamen inkl. Seiten- und Statusspalte. Aufnahme: Admin → Template → Felder extrahieren.}

\subsection{Felder gruppieren \& Reihenfolge festlegen}
\begin{enumerate}
  \item Hauptgruppen anlegen (z.\,B. Deutsch, Mathematik, Sozialverhalten).
  \item Felder per Auswahl/Batch zuordnen.
  \item Filter nutzen, um Varianten sauber zu trennen.
  \item Reihenfolge speichern und in der Lehreransicht testen.
\end{enumerate}

\HintBox{Die Gruppen bestimmen UI-Reihenfolge und Exportstruktur. Klare Benennungen helfen Lehrkräften, fachlich durch die Klassen zu navigieren, ohne zusätzliche Erläuterungen zu benötigen.}

\PlaceholderFigure{Abb. 5 – Gruppierung \& Filter}{Platzhalter: Gruppenliste mit Batch-Zuordnung, Filtern und Vorschau. Aufnahme: Admin → Template-Felder.}

\subsection{Optionslisten erstellen (Noten/Skalen)}\index{Optionsliste}
\begin{enumerate}
  \item Admin \(\rightarrow\) Optionslisten \(\rightarrow\) Neue Liste.
  \item Name eindeutig wählen (z.\,B. \enquote{Skala 1--5}).
  \item Einträge anlegen: technischer Value (stabil) + Label (Anzeige).
  \item Speichern und testen.
\end{enumerate}

\HintBox{Labels dürfen angepasst werden. Values bleiben stabil. Bei inhaltlichen Änderungen bitte eine neue Optionsliste anlegen oder versionieren.}

\PlaceholderFigure{Abb. 6 – Optionslisten-Editor}{Platzhalter: Editor mit Namensfeld und Tabelle der Values/Labels. Aufnahme: Admin → Optionslisten.}

\subsection{Optionslisten den Feldern zuordnen}\index{Optionsliste}
\begin{enumerate}
  \item Feld öffnen und Feldtyp Option/Note/Skala setzen.
  \item Passende Optionsliste auswählen.
  \item Batch-Zuordnung nutzen, wenn mehrere Felder identisch sind.
\end{enumerate}

\WarnBox{Ändern Sie keine bestehenden Values in produktiven Optionslisten. Wenn Skalen angepasst werden müssen, legen Sie eine neue Liste an und ordnen Sie sie allen betroffenen Feldern neu zu.}

\PlaceholderFigure{Abb. 7 – Zuordnung Optionsliste → Feld}{Platzhalter: Feldbearbeitung mit Optionslisten-Auswahl und Speichern-Button. Aufnahme: Admin → Template-Felder → Feld bearbeiten.}

\subsection*{Änderungen im laufenden Zeugnisprozess \textendash{} Warnhinweise}
\WarnBox{Nach Beginn der Eingaben sollten Sie keine Template-Wechsel, erneute Feldextraktionen oder Value-Änderungen in Optionslisten vornehmen. Dies kann zu Zuordnungsverlusten oder fehlerhaften Exporten führen.}
\textbf{Vorgehen, falls Änderungen unvermeidbar sind:}
\begin{itemize}
  \item Template oder Feldreihenfolge nur vor Start des Schreibens anpassen; bei zwingenden Änderungen zuerst Testklasse nutzen und Export prüfen.
  \item Optionslisten bei inhaltlichen Änderungen versionieren statt bestehende Values umzuschreiben.
  \item Bei Unsicherheit zunächst einen Einzel-Schüler-Export testen, bevor Klassenexporte ausgeführt werden.
\end{itemize}

\subsection{Klassen anlegen \& Template zuweisen}
\begin{enumerate}
  \item Admin \(\rightarrow\) Klassen \(\rightarrow\) Neu.
  \item Schuljahr, Stufe, Bezeichnung (z.\,B. 4a) eintragen.
  \item Template auswählen und Klasse aktiv setzen.
\end{enumerate}

\HintBox{Änderungen am Template wirken sofort auf alle zugeordneten Klassen. Prüfen Sie daher neue Vorlagen zunächst an einer Testklasse, bevor Sie produktive Klassen umstellen.}

\PlaceholderFigure{Abb. 8 – Klasse anlegen}{Platzhalter: Klassenformular mit Feldern für Schuljahr, Stufe, Bezeichnung und Template-Auswahl. Aufnahme: Admin → Klassen.}

\subsection{Lehrkräfte \& Schüler:innen verwalten}
\textbf{Lehrkräfte}
\begin{itemize}
  \item Lehrkraft anlegen (Name, Login, Passwort).
  \item Lehrkraft den Klassen zuordnen.
\end{itemize}

\textbf{Schüler:innen}
\begin{itemize}
  \item Schüler:innen anlegen oder importieren (Vorname, Nachname, Klasse).
  \item Schüler-Login vergeben oder erzeugen; Login-Liste für die Klasse ausgeben.
\end{itemize}

\subsection{Admin-Dashboard}
Das Dashboard zeigt Systemstatus, Bearbeitungsstände und Schnellzugriffe.
\begin{itemize}
  \item Überblick über aktive Klassen, zuletzt bearbeitete Templates und offene Delegationen.
  \item Hinweise auf fehlende Zuordnungen (z.\,B. Klasse ohne Template) oder offene Exporte.
  \item Quicklinks zu häufigen Aufgaben wie Optionslisten, Klassenverwaltung oder Export.
\end{itemize}

\subsection{Einstellungen}
Systemweite Parameter werden hier gepflegt.
\begin{itemize}
  \item Sprachen, Formatierungen und verfügbare Features (z.\,B. Elternzugang) aktivieren oder deaktivieren.
  \item Export- und Ablagepfade prüfen; Sicherstellen, dass Schreibrechte für den PDF-Export vorhanden sind.
  \item Rollenrechte nach Bedarf anpassen, ohne die fachliche Reihenfolge zu verletzen.
\end{itemize}

\HintBox{Änderungen in den Einstellungen sollten dokumentiert und im Kollegium kommuniziert werden, damit Abläufe konsistent bleiben.}

\subsection{Audit-Log}
Zur Nachvollziehbarkeit wichtiger Aktionen.
\begin{itemize}
  \item Protokolliert relevante Änderungen (z.\,B. Template-Wechsel, Nutzeranlage, Exporte).
  \item Filter- und Suchoptionen nutzen, um Ereignisse nach Datum oder Nutzer zu finden.
  \item Bei sicherheitsrelevanten Vorfällen zeitnah sichern und auswerten.
\end{itemize}

\PlaceholderFigure{Abb. 9 – Benutzer \& Zuordnungen}{Platzhalter: Benutzerliste und Zuordnungsdialog für Lehrkräfte und Schüler:innen. Aufnahme: Admin → Benutzerverwaltung oder Klasse → Zuordnung.}

\section{Lehrkräfte: Berichte erstellen}
Einstieg für Lehrkräfte: Klasse wählen, klassenweite Felder ausfüllen, Schülerberichte ergänzen, Freigabe steuern, Export durchführen. Die UI orientiert sich an den Gruppen aus dem Template; Pflichtfelder sind gekennzeichnet und Fortschrittsanzeigen zeigen fehlende Eingaben.

\subsection{Klasse auswählen \& Überblick}
\begin{enumerate}
  \item Anmelden im Lehrerbereich.
  \item Klasse wählen und Status (Freigabe, Bearbeitungsstand, Export) prüfen.
\end{enumerate}

\begin{tabular}{p{0.35\textwidth}p{0.6\textwidth}}
  \textbf{Statusfelder} & \textbf{Bedeutung}\\\hline
  Freigabe & Gibt an, ob Schüler:innen aktuell schreiben dürfen.\\
  Bearbeitungsstand & Zeigt, welche Gruppen oder Pflichtfelder noch offen sind.\\
  Export & Markiert, ob ein Testexport durchgeführt wurde.\\
\end{tabular}

\PlaceholderFigure{Abb. 10 – Lehrerbereich – Übersicht}{Platzhalter: Klassenliste/Startansicht mit Statusanzeigen. Aufnahme: Lehrerbereich nach Login.}

\subsection*{Bearbeitungsstand \& Statusanzeigen}
\begin{itemize}
  \item Fortschrittsbalken oder Listen zeigen offene Pflichtfelder pro Gruppe; \enquote{unvollständig} bedeutet meist fehlende Pflichtfelder oder nicht gespeicherte Eingaben.
  \item Empfohlene Kontrolle: vor Export mindestens eine Gruppe pro Klasse stichprobenartig prüfen und einen Einzel-Schüler-Export durchführen.
  \item Delegationen berücksichtigen: abgeschlossene Delegationen prüfen und zurückholen, falls Änderungen nötig sind.
\end{itemize}

\subsection{Klassenweite Felder (einmal \textendash{} für alle)}\index{Klassenfeld}
\begin{enumerate}
  \item Bereich \enquote{Klassenfelder} öffnen.
  \item Felder vollständig ausfüllen und speichern.
\end{enumerate}

\HintBox{Klassenweite Felder sollten früh ausgefüllt werden, da sie als Basis in individuellen Texten erscheinen können. Nutzen Sie Zwischenstände und speichern Sie nach jeder Gruppe.}

\PlaceholderFigure{Abb. 11 – Lehrerbereich – Klassenfelder}{Platzhalter: Bereich \enquote{Klassenfelder} mit Eingabefeldern und Speicher-Button. Aufnahme: Lehrerbereich → Klasse → Klassenfelder.}

\subsection{Schülerberichte (individuell)}
\begin{enumerate}
  \item Schüler auswählen.
  \item Fach/Gruppe wählen (z.\,B. Deutsch).
  \item Textfelder ausfüllen, Optionsfelder setzen, speichern.
\end{enumerate}

\HintBox{Zeit sparen durch Arbeiten \enquote{Fach für Fach}: pro Gruppe nacheinander alle Schüler:innen ausfüllen, statt pro Kind den gesamten Bericht.}

\PlaceholderFigure{Abb. 12 – Lehrerbereich – Schülerbericht}{Platzhalter: Einzelansicht eines Schülers mit Gruppenregister und Eingabefeldern. Aufnahme: Lehrerbereich → Klasse → Schüler.}

\subsection*{Delegationen und Teamarbeit}
\begin{itemize}
  \item Fachbereiche oder Gruppen können an Kolleg:innen delegiert werden; delegierte Bereiche erscheinen im Posteingang der Empfängerin.
  \item Statuspfad: offen \(\rightarrow\) in Bearbeitung \(\rightarrow\) abgeschlossen. Status kann bei Bedarf zurückgesetzt werden.
  \item Delegationen sind klassen- und gruppenbezogen, nicht schülerbezogen; sie entlasten beim arbeitsteiligen Schreiben.
\end{itemize}

\subsection{Schüler-Eingabe freigeben (kontrolliert)}\index{Freigabe}
\begin{enumerate}
  \item Freigabe aktivieren und Zeitfenster festlegen.
  \item Hinweistext für Schüler:innen hinterlegen (kurz, klar).
  \item Nach Ablauf Freigabe schließen; danach kein Schüler-Schreibzugriff.
\end{enumerate}

\PlaceholderFigure{Abb. 13 – Lehrerbereich – Freigabe}{Platzhalter: Freigabe-Schalter mit Zeitfenster und Hinweistext. Aufnahme: Lehrerbereich → Klasse → Freigabe.}

\section{Schüler:innen: Eingaben machen}
Schüler:innen können nur schreiben, wenn die Lehrkraft die Eingabe freigegeben hat. Standardzustand ist gesperrt.

\begin{itemize}
  \item Zugriff erfolgt je nach Schule per QR-Code oder Login-Daten; es werden ausschließlich eigene Felder angezeigt.
  \item Optional kann eine Vorlesefunktion (TTS) aktiv sein; prüfen Sie Browser-Kompatibilität im Vorfeld.
  \item Bei Fragen oder Unsicherheiten sollen Schüler:innen die Lehrkraft ansprechen, bevor sie absenden.
\end{itemize}

\subsection{Gesperrt vs. Freigegeben}
\PlaceholderFigure{Abb. 14 – Schülerbereich – gesperrt}{Platzhalter: Hinweisseite ohne Eingabefelder, weil Freigabe fehlt. Aufnahme: Schülerbereich ohne Freigabe.}

\PlaceholderFigure{Abb. 15 – Schülerbereich – freigegeben}{Platzhalter: Eingabemaske mit Feldern und Speichern-Button. Aufnahme: Schülerbereich mit aktiver Freigabe.}

\HintBox{Schreibe 3--5 Sätze, nenne mindestens ein Beispiel, bleib freundlich und ehrlich, speichere am Ende.}

\section{Elternzugang (optional)}
Falls freigeschaltet, können Eltern fertige Berichte lesen und ggf. Feedback geben.
\begin{itemize}
  \item Eltern erhalten einen Link oder Zugangscode, der nur Lesezugriff auf den Bericht ermöglicht; Schreibrechte bestehen nicht.
  \item Optionales Feedback-Feld kann aktiviert werden; Rückmeldungen sind für Lehrkräfte sichtbar. \textbf{TODO: Schulinterne Regelung zur Nutzung des Feedbacks ergänzen.}
  \item Datenschutz beachten: Links nicht öffentlich teilen; idealerweise mit Ablaufdatum oder Schutzmechanismus ausliefern.
\end{itemize}

\PlaceholderFigure{Abb. 16 – Elternansicht (Platzhalter)}{Platzhalter: Ansicht mit fertigem Bericht und optionalem Feedback-Feld. Aufnahme: Elternzugang nach Freischaltung.}

\section{PDF-Export \& Archivierung}\index{Export}
Alle Exporte basieren auf Template + gespeicherten Daten; die PDF bleibt führend.

\begin{itemize}
  \item Lehrkräfte exportieren auf Klassenebene oder testen zunächst Einzel-Schüler-Exporte, um Format und Inhalte zu prüfen.
  \item Admins können klassenübergreifend nach Schuljahr oder Klasse filtern; Exportrechte folgen dem Rollenmodell.
  \item Benennen Sie Exportpakete nachvollziehbar (z.\,B. \enquote{2025-LEB-4a.pdf}) und sichern Sie sie im Schularchiv.
\end{itemize}

\subsection{Export durch Lehrkräfte}
\begin{enumerate}
  \item Exportbereich öffnen.
  \item Test: einen Schüler exportieren und prüfen.
  \item Klassenexport durchführen und PDF stichprobenartig kontrollieren.
\end{enumerate}

\PlaceholderFigure{Abb. 17 – Export im Lehrerbereich}{Platzhalter: Exportbereich mit Einzel- und Klassenexport. Aufnahme: Lehrerbereich → Export.}

\subsection{Export durch Admin (alle Klassen)}
\begin{enumerate}
  \item Admin \(\rightarrow\) Export/Reports öffnen.
  \item Filter setzen (z.\,B. Schuljahr oder Klasse).
  \item Gesamtexport starten.
\end{enumerate}

\PlaceholderFigure{Abb. 18 – Export im Adminbereich}{Platzhalter: Admin-Export mit Filtern (Schuljahr/Klasse) und Start-Button. Aufnahme: Adminbereich → Export.}

\section{Rollenabgrenzung im Betrieb (Admin vs. Lehrkraft)}
\begin{itemize}
  \item \textbf{Admin im laufenden Betrieb:} Keine Template-Wechsel oder Feldextraktionen während der Schreibphase; Fokus auf technischer Unterstützung und Nutzerverwaltung. \textbf{TODO: Interne Eskalationswege bei dringenden Änderungen ergänzen.}
  \item \textbf{Lehrkraft:} Inhalte pflegen, Freigaben steuern, Exporte anstoßen; keine Änderungen an Templates oder Optionslisten.
  \item \textbf{Abstimmung:} Vor größeren Änderungen Kollegium informieren und Testlauf (Einzel-Schüler-Export) durchführen.
\end{itemize}

\section{Bearbeitungsstand \& Statusanzeigen (Übersicht)}
\begin{itemize}
  \item \textbf{Offene Pflichtfelder:} Markierungen oder Fortschrittsanzeigen weisen auf fehlende Eingaben hin; erst bei \enquote{vollständig} exportieren.
  \item \textbf{Freigabe-Status:} Klar erkennbar, ob Schüler:innen schreiben dürfen; nach Abschluss deaktivieren.
  \item \textbf{Qualitätssicherung:} Stichprobe (mindestens zwei Schüler:innen) und Testexport pro Klasse vor dem finalen Klassenexport.
\end{itemize}

\section{Nach dem Zeugnis: Archivierung \& Vorbereitung neues Schuljahr}
\begin{enumerate}
  \item Freigaben schließen und sicherstellen, dass keine Schüler-Eingabe mehr möglich ist.
  \item Exportdateien nachvollziehbar benennen (Schuljahr, Klasse) und im Schularchiv ablegen.
  \item Template oder Optionslisten erst nach Abschluss anpassen; Änderungen dokumentieren.
  \item Optional: Klassenübergang oder Schuljahrwechsel vorbereiten (z.\,B. neue Klassen anlegen). \textbf{TODO: Schuljahreswechsel-Prozess des Systems ergänzen.}
\end{enumerate}

\section{Support \& Eskalation (schulintern)}
\begin{itemize}
  \item \textbf{Erste Anlaufstelle:} Klassenlehrkraft oder zuständige:r Admin. \textbf{TODO: Konkrete Kontaktstellen (IT/Admin/Entwicklung) eintragen.}
  \item \textbf{Bei Fehlermeldungen mitschicken:} Screenshot, Uhrzeit, Nutzerrolle, Klasse, ausgeführter Schritt, verwendeter Browser.
  \item \textbf{Dringende Fälle:} Exportstopps oder Datenverlust umgehend an Admin/IT eskalieren; Änderungen dokumentieren.
\end{itemize}

\section{FAQ \& Troubleshooting}
\begin{longtable}{|p{0.45\textwidth}|p{0.45\textwidth}|}
\hline
\textbf{Problem} & \textbf{Lösung}\\
\hline
Dropdown zeigt keine Werte & Optionsliste fehlt oder wurde dem Feld nicht zugeordnet; Liste prüfen und zuweisen.\\
\hline
Schüler:innen sehen keine Eingabe & Freigabe durch Lehrkraft fehlt oder wurde geschlossen; Freigabe erneut aktivieren.\\
\hline
Felder erscheinen auf falschen Seiten & Felder erneut extrahieren und Seitenzuordnung prüfen.\\
\hline
Nach Optionslisten-Änderung stimmen Werte nicht & Values wurden geändert; neue Optionsliste anlegen oder Version wählen, Values stabil halten.\\
\hline
Admin-Export leer & Prüfen, ob Klassen Templates zugeordnet sind und ob die Exportrolle korrekt ist.\\
\hline
\end{longtable}

\HintBox{Checkliste vor dem Zeugnisdruck: Klassenfelder vollständig, zwei Schülerberichte stichprobenartig geprüft, Schüler-Freigabe geschlossen, PDF-Export (ein Schüler + Klasse) getestet, Ablage-/Archivordner vorbereitet.}

\appendix
\section{Screenshot-Checkliste}
Dieser Anhang hilft beim Ersetzen der Platzhalter-Abbildungen. Empfehlung: Browser auf 100\% Zoom, neutrale Testdaten und konsistente Fenstergröße nutzen.

\begin{tabularx}{\textwidth}{|c|X|X|}
\hline
\textbf{Abb.} & \textbf{Titel} & \textbf{Ort (wo aufnehmen)}\\
\hline
1 & Login / Rollen-Einstieg & Startseite / Login\\
\hline
2 & Installationsmaske & \texttt{install.php}\\
\hline
3 & Template-Upload & Admin \(\rightarrow\) Templates\\
\hline
4 & Feldliste nach Extraktion & Admin \(\rightarrow\) Template \(\rightarrow\) Felder extrahieren\\
\hline
5 & Gruppierung \& Filter & Admin \(\rightarrow\) Template-Felder\\
\hline
6 & Optionslisten-Editor & Admin \(\rightarrow\) Optionslisten\\
\hline
7 & Zuordnung Optionsliste \(\rightarrow\) Feld & Admin \(\rightarrow\) Template-Felder \(\rightarrow\) Feld bearbeiten\\
\hline
8 & Klasse anlegen & Admin \(\rightarrow\) Klassen\\
\hline
9 & Benutzer \& Zuordnungen & Admin \(\rightarrow\) Benutzer / Klasse \(\rightarrow\) Zuordnungen\\
\hline
10 & Lehrerbereich \textendash{} Übersicht & Lehrerbereich nach Login\\
\hline
11 & Lehrerbereich \textendash{} Klassenfelder & Lehrerbereich \(\rightarrow\) Klasse \(\rightarrow\) Klassenfelder\\
\hline
12 & Lehrerbereich \textendash{} Schülerbericht & Lehrerbereich \(\rightarrow\) Klasse \(\rightarrow\) Schüler\\
\hline
13 & Lehrerbereich \textendash{} Freigabe & Lehrerbereich \(\rightarrow\) Klasse \(\rightarrow\) Freigabe\\
\hline
14 & Schülerbereich \textendash{} gesperrt & Schülerbereich ohne Freigabe\\
\hline
15 & Schülerbereich \textendash{} freigegeben & Schülerbereich mit Freigabe\\
\hline
16 & Elternansicht (optional) & Elternzugang nach Freischaltung\\
\hline
17 & Export im Lehrerbereich & Lehrerbereich \(\rightarrow\) Export\\
\hline
18 & Export im Adminbereich & Adminbereich \(\rightarrow\) Export\\
\hline
\end{tabularx}

\section{Platzhalter-Screenshots}
\PlaceholderFigure{Admin-Dashboard Übersicht}{Admin-Dashboard (Platzhalter)}

\PlaceholderFigure{Lehrer: Bericht bearbeiten}{Lehrer-Eingabemaske (Platzhalter)}

\PlaceholderFigure{Delegations-Inbox}{Delegations-Inbox (Platzhalter)}

\PlaceholderFigure{Schülerzugang mit Vorlesefunktion}{Schüler-QR-Login und TTS-Bar (Platzhalter)}

\clearpage
\section{Schnellanleitung für Lehrkräfte}
\textbf{Ziel:} Innerhalb weniger Minuten Berichte pro Klasse pflegen.

\begin{enumerate}
  \item Einloggen unter \textbf{\url{TODO: Lehrer-URL einfügen}}.
  \item Klasse oder delegierten Fachbereich öffnen.
  \item Schüler auswählen \(\rightarrow\) Pflichtfelder und Freitextfelder ausfüllen.
  \item Fortschritt prüfen (Anzeige fehlender Felder beachten).
  \item Bei Bedarf delegieren: Fachbereiche an Kolleg:innen abgeben.
  \item Delegationen verwalten: Fachbereiche an Kolleg:innen delegieren oder zurückholen.
  \item Optional Schülerfreigabe setzen und nach Ablauf wieder schließen.
  \item Export je Schüler oder gesamte Klasse durchführen (PDF prüfen).
\end{enumerate}

\vspace{1cm}
{\small Stand: automatisch aus aktuellem Code abgeleitet. Bitte Anpassungen nach Software-Updates prüfen.}

\clearpage
\section{Mini-Glossar}
\begin{tabularx}{\textwidth}{|l|X|}
\hline
\textbf{Begriff} & \textbf{Kurzdefinition}\\
\hline
Template (PDF) & Vorlage mit Formularfeldern; ist die führende Instanz für Struktur und Export.\\
\hline
Feldextraktion & Auslesen der Formularfelder aus dem Template; liefert Feldnamen, Typ, Seite.\\
\hline
Optionsliste & Werteliste mit stabilem Value und änderbarem Label; wird Dropdown-Feldern zugeordnet.\\
\hline
Klassenfeld & Feld, das einmal je Klasse ausgefüllt wird und für alle Schüler:innen gilt.\\
\hline
Freigabe & Zeitlich begrenzte Schreibfreigabe für Schüler:innen, gesteuert durch die Lehrkraft.\\
\hline
Export & Erzeugung druckfertiger PDFs auf Basis von Template und gespeicherten Eingaben.\\
\hline
\end{tabularx}

\clearpage
\phantomsection
\addcontentsline{toc}{section}{Index}
\printindex

\end{document}
